% !TeX spellcheck = en_US
\documentclass[parskip=full]{report}

\usepackage{amsmath}
\usepackage{listings}
%\usepackage{beramono}
\usepackage{float}
\usepackage[utf8]{inputenc}
\usepackage[T1]{fontenc}
\usepackage{xcolor}
\usepackage[a4paper, margin={3cm}]{geometry}
\usepackage{hyperref}
\usepackage{graphicx}
\usepackage{svg}
\usepackage{subcaption}
\usepackage{float}
\usepackage{pdfpages}

\usepackage{tikz}

\usepackage{hyphenat}
\usepackage[english]{babel}
% Carattere monospaziato di default
\renewcommand{\ttdefault}{pcr}

\tikzstyle{block} = [draw, fill=blue!20, rectangle, 
minimum height=3em, minimum width=6em]
\tikzstyle{sum} = [draw, fill=blue!20, circle, node distance=1cm]
\tikzstyle{input} = [coordinate]
\tikzstyle{output} = [coordinate]
\tikzstyle{pinstyle} = [pin edge={to-,thin,black}]

\lstset{
	% wrap long lines on new line
	postbreak=\mbox{\textcolor{red}{$\hookrightarrow$}\space},
	breaklines=true, 
	columns=fullflexible,
	% tab and fonts
	tabsize=2,
	basicstyle=\ttfamily\small,
	% theme
	numbers=left,
	rulecolor=\color{black!30},	
	% UTF8 and escape
	escapeinside={\%TEX}{\^^M},
	inputencoding=utf8,
	extendedchars=true,
	literate={á}{{\'a}}1 {à}{{\`a}}1 {é}{{\'e}}1 {è}{{\`e}}1,
}


% Title Page
\title{
	\includegraphics[width=0.333\textwidth]{assets/unipi1.png} \\
	\textsc{University of Pisa} \\
	\vspace{.5cm}
	Artificial Intelligence and Data Engineering \\
	Business and Project Management \\
	\vspace{2cm}
	{\huge Analyzing and Enhancing \textit{Product Summaries} from Amazon Reviews Using ChatGPT API: A Parameter-based Approach}
	\vspace{2cm}
}

\author{
	\begin{tabular}{lr}
		Ricky Marinsalda \\
		Vittoria Acampora
	\end{tabular}
}


\begin{document}
\maketitle
\tableofcontents


\chapter{Introduction}


\chapter{Architecture}

We have the following sensors:

\begin{itemize}
	\item \textbf{float} sensors, used to monitor the water level
	\item \textbf{co2} sensors, used to monitor the co2 level
	\item \textbf{humidity}, used to monitor the environment's humidity
\end{itemize}

and the following actuators:



\section{Sensors}




\section{Data encoding}


\chapter{Analytics}

\section{Database}

We're using a \textit{MySQL}-compatibile database management system, cllaed \textit{MariaDB}. We're storing historic sensors' data in tables, one for each sensor's class, in the form of tuples: timestamp, sensor's id and sensor's datum. The following \textit{DDL} was used to build the zoo's database:


\section{Grafana}

Our project incorporates a real-time monitoring and visualization aspect through the implementation of a 


\chapter{Implementation's Details}
\section{Collector}



\section{IoT Nodes}

	

	

		

		
		This code represents a system for managing actuators in an Internet of Things (IoT) environment. Let's break down the code and explain its functionality:
		
		\subsection{The \texttt{Actuator} Interface}
		
		The \texttt{Actuator} interface defines the contract for an actuator, which is responsible for sending 
		
		\subsection{The \texttt{ActuatorManager} Interface}
	
		
		\subsection{The \texttt{Fan} Class}
		
		The \texttt{Fan} class implements the \texttt{Actuator} interface and represents a specific type of actuator for a fan. It includes a constructor that takes an IP address as a parameter and initializes a CoAP client for communication.
	
		
	


\end{document}          
